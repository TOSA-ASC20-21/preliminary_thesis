


\documentclass[a4paper, 11pt]{article}
\usepackage[UTF8]{ctex}
\usepackage{amsmath}
\usepackage{amsfonts}
\usepackage{amssymb}
\usepackage{apacite}
\usepackage{siunitx}
\usepackage{graphicx}
\usepackage{url}
\usepackage{subfigure}
\usepackage{float}
\usepackage{booktabs}
\usepackage{caption}


	\title{ASC Student Supercomputer Challenge 2020-2021 Preliminary Round Thesis}
	\author{TOSA ASC}
	\date{2020-12-30}


\begin{document}

	\maketitle
	\captionsetup[figure]{labelfont={bf},name={Photo.},labelsep=period}
	
	\begin{abstract}
		[Abstract here. 120 words]
	\end{abstract}
	
	\pagenumbering{roman}
	\tableofcontents
	\newpage
	\pagenumbering{arabic}
	
	

	\section{Introduction of the university department activities in supercomputing}
		\subsection{Supercomputing-related hardware \& software platforms} [Draft Here.]
		\subsection{Supercomputing-related courses \& interest groups} 
		
			Because of supercomputing are newly sprouted things, our school don’t have related courses or training, but we truly have a student club called Tongji Open Source Association. Committed to broaden the horizons of college students, especially for those who is interested in computer knowledge, TOSA give lectures about computer basic knowledge and advanced knowledge frequently.
			
		\subsection{Supercomputing-related research and applications}
		
			Unfortunately, our school's research on supercomputing has just started, and there is no mature research or applications. But we believe that the greatest thing must have the smallest begin. Now, more and more student joined our club and learn about these knowledge, interest will eventually turns into achievement.
			
		\subsection{Key achievements on supercomputing research} [Draft Here.]
	\section{Team introduction}
		\subsection{Team setup}
		
			Both of our teammates are interested in supercomputing and are from the same club called Tongji Open Source Association. Once we heard the competition, we immediately attention on it. After looking up relevant information and competition introduction on the Internet, we all hope to join in the competition. After several talk on social platform, our team leader select us according to our personal characteristics.
		
		\subsection{Team members} 
			
			\begin{figure}[H]
				\centering
				\includegraphics[width=0.3\textwidth]{F1.jpg}
				\caption{Lu Yan} %图片标题
				\label{fig:1}  %图片交叉引用时的标签
			\end{figure} 
			\paragraph{Lu Yan}: intrested in Computational Neuroscience and Neuromorphic Engineering.
			
			\paragraph{Liu Kongyang}: interested in Computer Graphics, Computer Vision and Machine Learning.
			
			\paragraph{Qiu Zili}: interested in Hardware Configuration and Maintainence of HPC.
			
			\paragraph{Jiang Wenyuan}: majors in Bioinformatics, and is interested in HPC's application in Biology.
			
			\paragraph{Xia Wenyong}: interested in algorithm designation of HPC.

		\subsection{Team motto} 
		
			The ASC20-21 Team of Tongji University is a brand new one, though we are neither experienced nor with adequate resources, all of us are talented in the art of problem solving, and is eager to embrace the new knowledge on HPC. We’ll try our best in each challenge!
		
		
	\section{Technical proposal requirements}
		[Draft Here.]
		\subsection{Design of HPC system} [Draft Here.]
		\subsection{HPL and HPCG} [Draft Here.]
		\subsection{Language Exam (LE) Challenge} [Draft Here.]
		\subsection{The QuEST Challenge} [Draft Here.]
		\subsection{The PRESTO Challenge} [Draft Here.]

	\section*{Attribution} [Attribution Here.]

\bibliographystyle{apacite}
\bibliography{ref}

\end{document}























